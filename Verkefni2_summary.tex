
\documentclass[12pt, a4paper]{article}
\usepackage[icelandic]{babel}
\usepackage[utf8]{inputenc}
\usepackage[T1]{fontenc}
\usepackage[left=1.5cm, right=1.5cm, top=2cm, bottom=1.5cm]{geometry}
\usepackage{amsmath}
\usepackage{amsthm}
\usepackage{enumerate}
\usepackage{amsfonts}
\usepackage{xparse}
%--------------------------------------------------------------------------------------------- Listings stillingar 
\usepackage{listings}
\usepackage{color}
\usepackage{etoolbox}
\usepackage{graphicx}
\definecolor{mygreen}{rgb}{0,0.7,0.42}
\definecolor{mygray}{rgb}{0.5,0.5,0.5}
\definecolor{mymauve}{rgb}{0.58,0,0.82}
\definecolor{backcolor}{rgb}{1.0, 0.98, 0.875}
\definecolor{harvestgold}{rgb}{0.85,0.57,0.0}
\definecolor{indigo}{rgb}{0.0,0.25,0.888888882}

\lstset{ %
        backgroundcolor=\color{backcolor},   % choose the background color; you must add \usepackage{color} or \usepackage{xcolor}
        basicstyle=\ttfamily\footnotesize,        % the size of the fonts that are used for the code
        breakatwhitespace=false,         % sets if automatic breaks should only happen at whitespace
        breaklines=true,                 % sets automatic line breaking
        captionpos=b,                    % sets the caption-position to bottom
        commentstyle=\color{mygreen},    % comment style
        deletekeywords={...},            % if you want to delete keywords from the given language
        escapeinside={\%*}{*)},          % if you want to add LaTeX within your code
        extendedchars=true,              % lets you use non-ASCII characters; for 8-bits encodings only, does not work with UTF-8
        frame=single,                          % adds a frame around the code
        keepspaces=true,                 % keeps spaces in text, useful for keeping indentation of code (possibly needs columns=flexible)
        keywordstyle=\color{indigo},       % keyword style
        language=Python,                 % the language of the code
        otherkeywords={*,...},            % if you want to add more keywords to the set
        numbers=left,                    % where to put the line-numbers; possible values are (none, left, right)
        numbersep=5pt,                   % how far the line-numbers are from the code
        numberstyle=\tiny\color{mygray}, % the style that is used for the line-numbers
        rulecolor=\color{black},         % if not set, the frame-color may be changed on line-breaks within not-black text (e.g. comments (green here))
        showspaces=false,                % show spaces everywhere adding particular underscores; it overrides 'showstringspaces'
        showstringspaces=false,          % underline spaces within strings only
        showtabs=false,                  % show tabs within strings adding particular underscores
        stepnumber=2,                    % the step between two line-numbers. If it's 1, each line will be numbered
        stringstyle=\color{mymauve},     % string literal style
        tabsize=2,                         % sets default tabsize to 2 spaces
        title=\lstname                   % show the filename of files included with \lstinputlisting; also try caption instead of title
}

\usepackage{fancyhdr}
 
\pagestyle{fancy}
\fancyhf{}
\rhead{Þorsteinn Hjörtur Jónsson - 13.09'15}
\lhead{Project I - Data Scraping}
\begin{document}

\section{Introduction}
This project's main focus and ambition was to develop tools to analyze data. The data in question was a set of measurements on air quality data for the year of 2014 to 2015 from the Grensásvegur 15 station.
The data was taken from the station's website and is registered at 30 minute intervals throughout the whole period, with some exceptions that might be due to instrument failure or something else.


With this in mind, a dashboard was created which consists of two streaming charts, one of which contained data on \(NO_2\) and \(H_2S\) levels while the other contained data on \(PM_{10}\) and \(PM_{2.5}\) levels.
These chemicals and particulates are thought to be correlated with traffic as they are produced by fossil fuel burning vehicles and spiked winter tyres. 
Seeing that the data was registered at such regular intervals it was decided to try and develop something which might resemble the effect of looking at the data while streaming. 
This was thought to be characteristic for the data and its main purpose as people vulnarable to bad air quality could benefit from having access to this kind of information.
Furthermore it was thought to be plausible that this kind of a tool might be useful for analyzing historical data.
Thus it was decided to make an attempt at creating a graphical animation out of air quality levels.  


Seeing that car traffic and winter tyres are suspected to be the main cause of bad air quality it was decided that an important question to answer would be whether this would be reflected 
as increased concentration of toxic levels depending on main traffic hours and winter time.
In order to answer this question each month in the data set was analyzed by looking at the mean value for each time of measurement throughout the whole month. 
That is all the measurements that take place at the same HH:MM throughout the month in question are combined in a subset of the data set from which the 
mean value was calculated. From this an average day throughout the month was obtained, and thus the possibility of looking at the date in relation to traffic hours and winter time with the same tool.


In order to highlight the traffic hours, radiation levels measured were used,as most humans usually prefer driving during the day it is often bright during traffic hours(at least there are rarely traffic jams during the night).
This was integrated into the graphics by making the background color of the animation a function of the radiation measured. A future extension of this tool could possibly integrate weather data in a similar way.


For the easiest comparison it was decided to put the ratio between the measured values and the severe health risk limit of measured quantities. 
Furthermore a horizontal line was drawn to mark the recommended health limit of measured quantities according to Icelandic authorities. 
These limits are different depending on authorities and thus this tool could potentially aid in making a decision when defining the proper health limits.
In fact it is surprising to see that the particulates levels are rather high throughout the whole year.


The tool was made so that it can be easily modified to take other variables into account or to animate anything else with respect to quantities and time. 
It provided a clear picture of the measured quantities of dangerous toxics that that are a health risk to the public and thus should be carefully monitored.
A further extesion to this tool could be to smoothen the plot and make the animation smoother throughout it's duration.


The Python packages Pandas, Datetime, dateutil and Matplotlib were the main packages that were used to develop the tool of analysis. Pandas is a very powerful when handling big data, matplotlib is the plotting package and
Datetime and dateutil provide very smart ways of manipulating data which is based on timeseries.




\pagebreak
\section{Code}
\lstinputlisting{va_animate.py}
\lstinputlisting{va_read_data.py}
\pagebreak
\end{document}
